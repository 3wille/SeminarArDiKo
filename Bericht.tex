\documentclass[1pt,a4paper,final]{article}
\usepackage[utf8]{inputenc}
\usepackage[german]{babel}
\usepackage[T1]{fontenc}
\usepackage{amsmath}
\usepackage{amsfonts}
\usepackage{amssymb}
\usepackage[backend=biber]{biblatex}
\usepackage{gensymb}
\usepackage{graphicx}
\usepackage[onehalfspacing]{setspace}
\usepackage{url}
\usepackage{acronym}
%\usepackage{geometry}
%\geometry{verbose,a4paper,tmargin=25mm,bmargin=25mm,lmargin=30mm,rmargin=30mm}
\addbibresource{Bericht.bib}
\setcounter{secnumdepth}{5}
\setcounter{tocdepth}{5}

\title{\LARGE \bf
Bla
}


\author{Frederik Wille, Julian Deinert}
\date{\today}

\begin{document}



%\maketitle
%\thispagestyle{empty}
%\pagestyle{empty}

\maketitle
\thispagestyle{empty}
\newpage
\thispagestyle{empty}
\tableofcontents
\newpage
\setcounter{page}{1}

\section*{Abkürzungen}
\begin{acronym}
		\acro{igp}[IGP]{Interiour Gateway Protocol}
		\acro{as}[AS]{Autonomous System}
		\acro{ospf}[OSPF]{Open Shortest Path First}
\end{acronym}
\newpage
%%%%%%%%%%%%%%%%%%%%%%%%%%%%%%%%%%%%%%%%%%%%%%%%%%%%%%%%%%%%%%%%%%%%%%%%%%%%%%%%
\begin{abstract}
ABSTRACT

\end{abstract}

\section{Einleitung}
Freddy
\section{Routing Algorithmen}
\subsection{IGP}
Freddy
\ac{igp} \ac{igp}
\subsection{Distance-Vector-Routing-Protokolle}
Freddy
\subsection{Link-State-Routing-Protokolle}
Link-State-Routing-Protokolle sind eine Klasse von Routing-Protokollen, die sich dadurch auszeichnen, dass sie ihre Informationen, über die verfügbaren Routen in ihrem Netzwerk, mit all ihren Nachbarn teilen. Im Gegensatz zu Distance-Vector-Routing-Protokollen senden sie mehrere kleine Aktualisierungen, die jeweils vollständige Routen zu andern Knoten des Netzwerks enthalten, anstatt nur direkte Verbindungen zu ihren Nachbarknoten zu übermitteln.
\subsubsection{Dijkstra-Algorithmus}
Julian
\section{Open Shortest Path First}
\ac{ospf} \ac{ospf}
\subsection{Areas}
\subsection{Router Typen}
\subsection{Metriken}
\section{Fazit}
Julian

\clearpage
\nocite{*}
\printbibliography
\end{document}
